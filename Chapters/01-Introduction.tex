



\chapter[课业相关]{课业相关}
\label{cp:introduction}

{
\parindent0pt

\textit{Author: Strtus}

\textit{Current Version: 2.2.1}


\vspace{.935em}

\indent 
\noindent\hspace{2em}本章节将对代培期间所涉及的课程类型、选课流程进行简要的介绍并给出用于参考的课程建议


\section{培养方案与学分结构}

\noindent\hspace{2em}硕士研究生课程体系包括学位课和非学位课,由公共必修学位课、专业必修学位课、专业选修学位课以及公共选修课(非学位)组成。在申请硕士学位前,需要修读不少于37个学分,其中,学位课程总学分不应低于35个学分,且所有学位课程加权平均分必须达到75分以上


\renewcommand{\arraystretch}{1.2}

\begin{table}[h!]
    \centering
    \begin{tabularx}{\textwidth}{|>{\centering\arraybackslash}X
                                 |>{\centering\arraybackslash}X
                                 |>{\centering\arraybackslash}X|}
        \hline
        \textbf{课程属性} & \textbf{类别} & \textbf{学分要求} \\
        \hline
        \multirow{3}{*}{学位课} 
            & 公共必修课 & 8学分 \\
        \cline{2-3}
            & 专业必修课 & 3--4门,10--12学分 \\
        \cline{2-3}
            & 专业选修课 & 4--7门,15--20学分 \\
        \hline
        非学位课 & 公共选修课 & 1--2门,$\geq$2学分 \\
        \hline
        课程总分 &  & $\geq$37学分(学位课$\geq$35学分) \\
        \hline
    \end{tabularx}
    \caption{学分结构概览}
    \label{tab:credit-requirements}
\end{table}

\noindent\hspace{2em}中科院的培养模式区别于传统高校,对于代培生并未设置统一的培养方案,因而在选课时拥有相当灵活的空间,但你仍然需要完成一些必修课程,参考下表


\begin{table}[htbp]
\centering
\renewcommand{\arraystretch}{1.4}
\begin{tabularx}{\textwidth}{|>{\centering\arraybackslash}p{3.8cm}|
                                  >{\centering\arraybackslash}X|
                                  >{\centering\arraybackslash}p{1.2cm}|
                                  >{\centering\arraybackslash}c|
                                  >{\centering\arraybackslash}c|
                                  >{\centering\arraybackslash}c|}
\hline
\textbf{课程属性} & \textbf{课程名称} & \textbf{学分} & \textbf{学硕} & \textbf{专硕} & \textbf{直博} \\
\hline
\multirow{8}{*}{公共课}
& 基础英语课程 & 2 & \checkmark & \checkmark & \checkmark \\
& 应用英语课程 & 2 & \checkmark & \checkmark & \checkmark \\
& 《科技论文写作》 & 2 & & & \checkmark \\
& 《自然辩证法概论》 & 1 & \checkmark & \checkmark & \checkmark \\
& 《中国特色社会主义理论》 & 2 & \checkmark & \checkmark & \checkmark \\
& 《马克思主义》 & 2 & & & \checkmark \\
& 公共选修课 & 2 & \checkmark & \checkmark & \checkmark \\
& 人文系列讲座 & 1 & \checkmark & \checkmark & \checkmark \\
\hline
\multirow{1}{*}{专业必修课} 
& 与导师协商确定 & 10--12 & \checkmark & \checkmark & \checkmark \\
\hline
\multirow{3}{*}{专业选修课} 
& 与导师协商确定 & 15--20 & \checkmark & \checkmark & \checkmark \\
& 《电子信息检索》 & 2 & \checkmark & \checkmark & \checkmark \\
& 《工程伦理》 & 2 & &\checkmark  & \\
& 《学术道德与写作规范》 & 2 & \checkmark & \checkmark & \checkmark \\
\hline
\end{tabularx}
\caption{课程结构与学分要求}
\end{table}

\begin{block}[note]
\textit
{1.【基础英语】也即《研究生综合英语》,满足条件可申请免修\newline
2.【应用英语】包含《日常交流英语》、《学术交流英语》、《科技论文写作》三门课,硕士选择选择任意一门即可,直博生则无法修读《科技论文写作》\newline
3.【人文系列讲座】为新增设的必修课程,这部分学分回所修\newline
4.课程结构与学分要求每年波动较大,本表格仅作为参考,如有疑问,请联系教学秘书进一步核实
}
\end{block}